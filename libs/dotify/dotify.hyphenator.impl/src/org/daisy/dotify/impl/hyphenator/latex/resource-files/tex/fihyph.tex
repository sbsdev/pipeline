%
% ----->  Finnish hyphenation patterns for MLPCTeX  <------
% First release January -86 by Kauko Saarinen,
% Computing Centre, University of Jyvaskyla, Finland
%
% Completely rewritten January -88. The new patterns make
% much less mistakes with foreign and compound words.
% The article "Automatic Hyphenation of Finnish"
% by Professor Fred Karlsson is also referred
% ---------------------------------------------------------
%
% 8th March -89 (vers. 2.2), some vowel triples by Fred Karlsson added.
%
% *********     Patterns may be freely distributed   **********
%
%
\patterns{
1ba
1be
1bi
1bo
1bu
1by
1da
1de
1di
1do
1du
1dy
1d^^e4
1d^^f6
1fa
1fe
1fi
1fo
1fu
1fy
1ga
1ge
1gi
1go
1gu
1gy
1g^^e4
1g^^f6
1ha
1he
1hi
1ho
1hu
1hy
1h^^e4
1h^^f6
1ja
1je
1ji
1jo
1ju
1jy
1j^^e4
1j^^f6
1ka
1ke
1ki
1ko
1ku
1ky
1k^^e4
1k^^f6
1la
1le
1li
1lo
1lu
1ly
1l^^e4
1l^^f6
1ma
1me
1mi
1mo
1mu
1my
1m^^e4
1m^^f6
1na
1ne
1ni
1no
1nu
1ny
1n^^e4
1n^^f6
1pa
1pe
1pi
1po
1pu
1py
1p^^e4
1p^^f6
1ra
1re
1ri
1ro
1ru
1ry
1r^^e4
1r^^f6
1sa
1se
1si
1so
1su
1sy
1s^^e4
1s^^f6
1ta
1te
1ti
1to
1tu
1ty
1t^^e4
1t^^f6
1va
1ve
1vi
1vo
1vu
1vy
1v^^e4
1v^^f6
% ------- Some common words borrowed from other languages -------
% ------- This part could be updated from time to time    -------
%
1st2r  % -stressi, -strategia etc.
%
%  ------ Some special cases occuring with compound words only ----
%  ------ There still remains well known problem as 'kaivos\-aukko' etc.
%a1y   (common in borrowed words)
^^e42y
y1a2
y1o2
o1y
^^f62y
u1y2
y1u2
^^f63a2
^^f63o2
^^e43a2
^^e43o2
^^e41u2
^^f61u2
a1^^e4    % (a1^^e42 ei mahdollinen!)
a1^^f6
o1^^e4
o1^^f6
u1^^e42
u1^^f62
^^e42^^e4
^^f62^^f6
^^e42^^f6
^^f62^^e4
%    lyhyt/pitka -vokaalipareja, tavallisesti sanarajalla
aa1i2    % maa-ikkuna
aa1e2
aa1o2    % maa-ottelu
aa1u2    % uraa-uurtava
ee1a2    % tee-astia
ee1i2    % tee-istutus
ee1u2    % varietee-uusinta
ee1y2
ii1a2
ii1e2
ii1o2
uu1a2
uu1e2    % puu-esine
uu1o2    % puu-osa
uu1i2    % puu-istutus
e1aa
i1aa
o1aa
u1aa
u1ee
a1uu     % kala-uuni
i1uu     % ravi-uutiset
e1uu     % virhe-uutinen
o1uu     % radio-uutiset
^^e4^^e41i
^^e4^^e41e
^^e4^^e43y
i1^^e4^^e4
e1^^e4^^e4
y1^^e4^^e4
i1^^f6^^f6  % yhti-^^f6^^f6n etc.
%i1eu     % keski-eurooppalainen
%          vokaalikolmikkoja etc.  yhdyssanojen rajoissa
% -------- vowel triples by Fred Karlsson
a1ei
a1oi
e1ai
% e1oi % ambiguous for ex. video-ilme (8.3.89)
i1au
% u1oi   % ambiguous    (8.3.89)
y1ei
ai1a
ai1e
ai1o
ai1u
au1a
au1e
eu1a
ie1a
ie1o
%ie1u   % ambiguous
ie1y
io1a2
io1e2
iu1a
iu1e
iu1o
oi1a
oi1e
oi1o
oi1u
o1ui % veto-uistin, himo-uimari   etc.
ou1e
ou1o
ue1a
ui1e
uo1a
% uo1i % ambiguous
uo1u
%   ---------------- End of vowel triples --------------------
e1^^f62
^^f61e2
.^^e42     % don't hyphenate  ^^e4-lyllinen etc.
%
% The following patterns contain no general scientific rule. They
% are selected more or less intuitively to solve problems
% with common and frequently appearing compound words.
% However, every pattern resolves more than only one
% hyphenation problem.
%
u2s         % estaa virheita yhdyssanojen yhteydessa
yli1o2p     % yli-opisto etc.
ali1a2v     % ali-avaruus etc.
1sp2li      % kuutio-splini etc.
alous1
keus1       % oikeus-oppinut etc.
rtaus1
2s1ohje     % -sohjelma   etc.
2s1a2sia    % nais-asianainen etc.
1a2sian     % neuvottelu-asian  etc.
1a2siat     % koti-asian  (ei kotia-sian)
1a2sioi     % talous-asioita etc.
r2as l2as   % tikku-rasia  etc
2s1o2pisk   % xs-opiskelija  etc
2n1o2pet
2s1a2loi
2n1o2pist    % kansan-opisto etc.
2s1o2pist
2s1o2sa      % xxs-osakas etc.
2n1o2sa      % asian-osainen etc.
alkei2s1
perus1
2s1i2dea. 2s1i2dean
2s1e2sity    % xs-esitys etc
2n1e2dus     % kansan-edustaja etc.
2s1ajatu     % -ajatus etc.
2s1ase
2s1apu
2s1y2rit     % yhteis-yritys etc.
.ydi2n1
.suu2r1a2    % suur-ajot etc.
2s1y2hti
2n1otto 2n1oton
2n1anto 2n1anno
2n1a2jan 2n1aika
2n1o2mai
2n1y2lit
2s1a2len
2n1a2len
1a2siaka2s1
ulo2s1        % ulos-ajo
2n1a2jo       % kiven-ajo
2s1a2jo
%
% *** The following rules may be used on user's responsibility ***
% *** for example, may be needed with narrow columns           ***
%       >>>>>>>>>>>  a1e a1o e1o o1a u1a  <<<<<<<<<<<
%
% ----- Some districting rules by Professor Fred Karlsson's ideas   ------
%
b2l 1b2lo bib3li
b2r 1b2ri 1b2ro 1b2ru
d2r 1d2ra
f2l 1f2la
f2r 1f2ra 1f2re
g2l 1g2lo
g2r 1g2ra
k2l
1k2ra 1k2re 1k2ri
1k2v 1k2va
p2l
p2r 1p2ro
c2l
q2v 1q2vi
sc2h ts2h
ch2r
}

